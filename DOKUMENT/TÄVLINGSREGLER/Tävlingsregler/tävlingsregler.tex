\documentclass{article}
\usepackage[utf8]{inputenc}

\title{Förslag på \\banspeficikation för labyrint och tävlingsregler\\ för undsättningsrobot}
\author{Alexander Eriksson, projektgrupp 4
\\  Ola Grankvist, projektgrupp 4
\\ Måns Skytt, projektgrupp 2
\\ Adnan Berberovic, projektgrupp 2}
\date{January 2015}

\begin{document}

\maketitle

\section{Introduktion}
Det här är ett förslag på det dokument som är tänkt att reglera utformandet av labyrinten samt tävlingsregerna för undsättningsrobotar i tävlingen i slutet av kursen TSEA56: Elektronikprojekt. 

\section{Banspecifikation}
Följande regler skall gälla för banan:
\begin{itemize}
\item Labyrintens hölje skall vara max 6x6m,
\item Labyrintens korridorer skall vara 0.4m breda, 
\item Väggsegment måste vara uppbyggda så att de innesluter en area på minst 0.4x0.4m,
\item Labyrintens korridorers hörn skall utgöras av vinkelräta väggelement, 
\item Mål skall vara 0.4x0.4m och markeras heltäckande svart,
\item Start skall vara en 0.4x0.4m heltäckande svart ruta, utanför de 6x6m som utgör labyrinten samt ha tre anslutande väggar,
\item Underlaget skall vara plant och likadant överallt,
\item Underlaget skall vara i skarp kontrast till målet,
\item Underlaget skall ge tillräckligt bra friktion mot robotarnas hjul för att glidning inte skall uppstå, 
\item Underlag skall inte kunna förflyttas av roboten, 
\item Samtliga väggelement skall vara av samma material, i skarp kontrast till målet samt 0.4m långa, 
\end{itemize}

\section{Tävlingsregler}
Robotens uppdrag är att från en startpunkt kartlägga en labyrint (enligt banspecifikation) tillräckligt bra för att kunna identifiera den kortaste vägen till målet. Roboten skall sedan återvända till startpunkten varefter den skall ta den kortaste vägent till målet, lämna "förnödenheter"  och sedan återvända. 
Följande regler skall gälla:
\begin{itemize}
\item Den kortaste vägen avser kortast på det redan utforskade området, hela labyrinten måste inte utforskas,
\item Uppdraget skall utföras tre gånger från tre olika startpositioner, alla robotar startar från samma startposition innan startpositionen byts,
\item Vid nytt uppdrag får roboten ej ha kännedom om labyrintens uppbyggnad utöver i banspecifikationen specificerade detaljer,
\item Roboten skall köra autonomt, 
\item Den robot vars totala körtid (för alla tre utföranden) utför uppdraget snabbast vinner, 
\item Roboten för köra max 10 minuter per utförande, 
\item Vid start placeras roboten valfritt i startrutan med alla delar innanför rutans kanter,
\item "Förnödenheterna" får, då roboten återvänt till den utmärkta startrutan, laddas manuellt av styrande grupp,
\item "Förnödenheterna" som skall transporteras skall vara en tom, intakt MER-förpackning,
\item "Förnödenheterna" skall placeras helt inom målområdet ståendes, 
\item Tjockleken på MER-förpackningen får ej avvika med mer än 1cm någonstans på förpackningen under hela utförandet, 
\item Vinnarlaget får MER-förpackning
\item Tävlingstiden startar då knappen på roboten som startar autonom styrning trycks ned och avslutas då roboten har återvänt och är stillastående i startpositionen,

\end{itemize}


\end{document}

\begin{comment}

\end{comment}