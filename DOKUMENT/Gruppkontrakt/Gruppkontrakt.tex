\documentclass[11pt]{article}
\usepackage[margin=3cm]{geometry}
\usepackage[swedish]{babel}
\usepackage[utf8]{inputenc}
\usepackage[T1]{fontenc}
\usepackage{fancyhdr}
\usepackage{ragged2e}
\usepackage{titling}
\usepackage{graphicx}
\usepackage{pbox}
\usepackage{tabularx}
\usepackage{longtable}
\usepackage{tabu}
\usepackage{url}


\pagestyle{fancy}


\date{}
\pagenumbering{roman}
\chead{Undsättningsrobot}
\rhead{2015-02-13}
\lhead{}
\lfoot{
	TSEA56 
	\\
	Gruppkontrakt
}
\rfoot{
	Projektgrupp 2
	\\
	e-post: adnbe196@student.liu.se
}

\begin{document}
\begin{titlepage}
\begin{center}
{\Large\bfseries TSEA56 - Kandidatprojekt i elektronik \\ Gruppkontrakt}\\
%
\vspace{2\baselineskip}
%
Version 0.1\\
\vspace{2\baselineskip}
%
Grupp 2 \\
Agafonov, Nikolaj, 
\texttt{nikag669}
\\
Berberovic, Adnan, 
\texttt{adnbe196}
\\
Brorsson, Andreas, 
\texttt{andbr981}
\\
Fridborn, Fredrik, 
\texttt{frefr166}
\\
Oprea, Robert, 
\texttt{robop806}
\\
Skytt, Måns, 
\texttt{mansk700}

\vspace{2\baselineskip}
2015-02-11


\end{center}
\end{titlepage}



\pagebreak

\pagenumbering{arabic}

\begin{flushleft}
\section{Syfte}
Syftet med att upprätta ett gruppkontrakt är att alla i gruppen ska vara överens om vilka arbetsformer som ska gälla. Upprättandet leder även till diskussioner och reflektion kring frågor som är viktiga för gruppens trivsel.

\section{Målet med projektet}
Vår grupps ambitionsnivå är att arbetet ska leda till att det framtagna resultatet i projektet blir det bästa tänkbara.


\section{Rutiner}

\begin{itemize}
  \item Alla deltagare i gruppen ska deltaga vid de tillfällen som gruppen kommit överrens om.
  
  \item Vid sammankomster ska gruppmedlemarna komma väl förbereda för vad som tas upp.
  
  \item Sammankommst sker när ansvarig gruppmedlem tycker det behövs, Samordnare (Projektledaren) ansvarar då för tid och plats när gruppen ska samkomsta.
  
\item Varje träff avslutas med en utvärdering, där var och en belyser hur arbetet i gruppen fungerat.  
  
  \item Varje gruppmedlem ska arbeta med de uppgifter som denne fått enligt projektplanen.
  
  \item Frånvaro ska meddelas i god tid eller om sjukdom uppstår.
  
  
\end{itemize}


\section{Kommunikation}
Grupp-komunikation sker genom:
\begin{itemize}

\item Facebook-chat

\item SMS eller telefon-samtal

\end{itemize}

Om någon slutar kominusera när den ska och inget skäl innan har givits tas detta upp på nästkommande möte. 

\section{Roller}

\begin{itemize}
\item Projektledare - Adnan
\item Dokumentansvarig/Sekriterare? - 
\item Hårdvaruansarig -  
\item Mjukvaruansvarig - 
\item blabla - 
\item blabla - 

\end{itemize}

\section{Resurser}
Hårdvarukomponenter får vi tillgång till från beställaren? Med möjlighet att även köpa in nya komponenter om sådant behövs och godkänds av beställare. Gruppen har tillgång till [Grupprum] dygnet runt under perioden från att xx är klar till projettiden är slut.
Gruppen har tillgång till lånedator och förvaringsskåp i [Grupprum].


\section{Beslutsformer}

\begin{itemize}
\item Målet är att nå konsensus men om det inte fungerar gäller majoritetsbeslut.

\item Vid lika beslut är det ledaren över den rörda områdets röst som väger tyngst


\end{itemize}


\section{Ansvar}
Alla gruppmedlemmar har ett gemensamt ansvar för att projektet utförs och blir klart. 

Varje gruppmedlem ansvarar för en roll, och om denna inte utförs enligt förväntningar tas detta upp på nästkommande möte.

Den som inte bidrar aktivt i gruppen ska inte heller dra nytta av gruppens gemensama resultat.

Det är viktigt att ge varandra återkoppling, såväl positiv som negativ.

Samarbetet i gruppen måste när som helst kunna diskuteras öppet, även om det innebär obehag för någon. 

\end{flushleft}



\end{document}