\documentclass[11pt]{article}
\usepackage[margin=3cm]{geometry}
\usepackage[swedish]{babel}
\usepackage[utf8]{inputenc}
\usepackage[T1]{fontenc}
\usepackage{fancyhdr}
\usepackage{ragged2e}
\usepackage{titling}
\usepackage{graphicx}
\usepackage{pbox}
\usepackage{tabularx}
\usepackage{longtable}
\usepackage{tabu}
\usepackage{url}

\graphicspath{ {images/} }
\newcommand{\subtitle}[1]{%
  \posttitle{%
    \par\end{center}
    \begin{center}\large#1\end{center}
    \vskip0.5em}%
}
\newcounter{kravc}
\setcounter{kravc}{1}
\newcommand{\kravcc}{
	\thekravc
	\stepcounter{kravc}
}
\pagestyle{fancy}


\date{}
\pagenumbering{roman}
\chead{Undsättningsrobot}
\rhead{2015-02-11}
\lhead{}
\lfoot{
	TSEA56 
	\\
	Projektplan
}
\rfoot{
	Projektgrupp 2
	\\
	e-post: adnbe196@student.liu.se
}

\begin{document}
\begin{titlepage}
\begin{center}
{\Large\bfseries TSEA56 - Kandidatprojekt i elektronik \\ LIPS Projektplan}\\
%
\vspace{2\baselineskip}
%
Version 0.1\\
\vspace{2\baselineskip}
%
Grupp 2 \\
Agafonov, Nikolaj, 
\texttt{nikag669}
\\
Berberovic, Adnan, 
\texttt{adnbe196}
\\
Brorsson, Andreas, 
\texttt{andbr981}
\\
Fridborn, Fredrik, 
\texttt{frefr166}
\\
Oprea, Robert, 
\texttt{robop806}
\\
Skytt, Måns, 
\texttt{mansk700}

\vspace{2\baselineskip}
2015-02-11

\vspace{25\baselineskip}
Status
\begin{longtable}{|l|l|l|} \hline

Granskad &
 & 
 \\ \hline
Godkänd &
 &
 \\ \hline
 
\end{longtable}

\end{center}
\end{titlepage}

\pagebreak
\begin{center}

\section*{PROJEKTIDENTITET}
2015/VT, Undsättningsrobot Gr. 2
\\
Linköpings tekniska högskola, ISY
\\[0.5in]
\begin{table}[h]
\begin{tabular}{|l|l|l|l|} \hline
Namn & Ansvar & Telefon & E-post \\[0.1in] \hline
Nikolaj Agafonov & Dokumentansvarig (DA) & 072-276 99 46 & nikag669@student.liu.se \\ \hline
Adnan Berberovic & Projektledare (PL) & 070-491 96 07 & adnbe196@student.liu.se \\ \hline
Andreas Brorsson & Leveransansvarig (LA) & 073-524 44 60 & andbr981@student.liu.se \\ \hline
Fredrik Fridborn & Designansvarig mjukvara (DAM) & 073-585 52 01 & frefr166@student.liu.se \\ \hline
Robert Oprea & Testansvarig (TA) & 070-022 10 18 & robop806@student.liu.se \\ \hline
Måns Skytt & Designansvarig hårdvara (DAH) & 070-354 28 84 & mansk700@student.liu.se \\ \hline
\end{tabular}
\end{table}

E-postlista för hela gruppen: adnbe196@student.liu.se
\\[1in]
Kund: Kent Palmkvist, 581 83 Linköping,
Kundtelefon: 013-28 13 47, kentp@isy.liu.se
\\[1in]
Kursansvarig: Tomas Svensson, 013-28 13 68, tomass@isy.liu.se
\\
Handledare: Olov Andersson, 013-28 26 58, Olov.Andersson@liu.se
\end{center}
\pagebreak

\tableofcontents

\pagebreak

\section*{Dokumenthistorik}
\begin{table}[h]
\begin{tabular}{|l|l|l|l|l|} \hline

Version & 
Datum & 
Utförda förändringar & 
Utförda av & 
Granskad \\[0.1in] \hline
0.1 &
2015-02-11 & 
Första utkastet & 
Grupp 2 & 
\\ \hline

\end{tabular}
\end{table}


\pagebreak

\pagenumbering{arabic}

\begin{flushleft}

\section{Beställare}
Projektets beställare är Kent Palmkvist som representerar ISY.
\pagebreak

\section{Översiktlig beskrivning av projektet}

\subsection{Syfte och mål}
Mål är att leverera en produkt, en robot, som kan köra autonomt och via fjärrstyrning i okända, möjligtvis farliga, miljöer. Dessutom ska projektet visa hur man tillämpar kunskap från de kurser man läst, samt ge erfarenhet i projektarbete och förståelse för hur hårdvara och mjukvara interagerar.
\subsection{Leveranser} %Lägga till aktiviteter?
Leveranser skall göras senast på nedan nämnda tider och datum om inte annat är överenskommet mellan beställare och projektgrupp.

%Tabell med leveranser
\begin{center}
\begin{longtable}{|l |p{.8\linewidth}|} \hline

3 feb: & 
kl 16.00: Kravspecifikationen ska vara klar. (BP1) \\ \hline

16 feb: & 
kl 16.00: Första versionen av projektplan, tidplan och systemskiss ska vara inlämnade till beställaren. \\ \hline

20 feb: & 
kl 16.00: Slutgiltig version av projektplan, tidplan och systemskiss ska vara inlämnade till beställaren. \\ \hline

5 mars: &
kl 16.00: första version av förstudien (minst 5 sidor) ska skickas till respektive handledare och till er beställare. \\ \hline

11 mars: & 
kl 16.00: Första versionen av designspecifikationen ska vara inlämnad till handledaren. \\ \hline

24 mars: &
Designspecifikationen ska vara godkänd av handledaren vid ett beslutsmöte BP3. \\ \hline

1 april: &
kl 16:00 Version 1.0 av förstudien ska skickas till respektive handledare och till beställare. \\ \hline

17 april: & 
Nuvarande design ska vara presenterad för och godkänd av handledaren vid ett beslutsmöte BP4. \\ \hline

25 maj: &
Verifiering av kraven (BP5) bör ske i god tid innan redovisningen. Utan detta beslut får ni inte leverera! \\ \hline

21 maj: &
Kappan, version 1.0, (exklusive appendix) ska levereras. Se nedan. \\ \hline

27 maj: &
Teknisk dokumentation och användarhandledning (båda version 1.0) ska vara inlämnade. Slutversion av skrivarbete skall också skickas med vid detta tillfälle. \\ \hline

Vecka 23: &
Redovisning och demonstration.\\ \hline

2 juni: &
(preliminärt) 8.15-17 muntliga presentationer och opposition. Tider se nedan. \\ \hline

3 juni: &
(preliminärt) 9.15-17 tävlingar utanför café Java. \\ \hline

5 juni: &
Efterstudien ska vara inlämnad. Vid denna tidpunkt ska även källkod skickas in i en zip-fil. \\ \hline

12 juni: &
Bärbar dator och övrig utrustning ska vara återlämnade. \\ \hline
\end{longtable}
\end{center}

En tidrapport ska lämnas senast kl 16.00 vid följande datum: 4 febr, 23 febr, 9 mars, 23 mars, 30 mars, 13 april, 20 april, 27 april, 4 maj, 11 maj, 18 maj, 25 maj, 1 juni och 8 juni.

\subsection{Begränsningar} %Vad behöver vi inte göra?
Miljön som roboten körs i är begränsad på så sätt att den max ska vara 6x6m och passager måste vara bredare än 40 cm. Inga blockerande hinder får förekomma i vägen för roboten, den kan endast röra sig på släta ytor.
\pagebreak


\section{Fasplan}
Nedan ges en grov beskrivning av aktiviteterna i varje fas.
\subsection{Under projektet}
Initialt kommer en stor del av arbetet bestå av att lära sig de verktyg som behövs för att kunna genomföra projektet: AVR, VDHL, dataöverföring via bluetooth, mätteknik etc. Det sker dels genom läsning men även laboratoriskt. Därefter kommer hårdvaran konstrueras och testning kommer ske kontinuerligt. Efter varje modul skapats kommer större tester genomföras. Slutligen kommer mjukvaran kodas med kontinuerlig testning.

\subsection{Efter projektet}
Efter projektet kommer efterstudie genomföras, labbplatsen städas och materiel återlämnas varpå projektgruppen upplöses och gruppkontraktet hävs.
\pagebreak

\section{Organisationsplan för hela projektet} 
%Gör en enkel organisationsplan (figur?).

\subsection{Villkor för samarbetet inom projektgruppen}
Samarbetet inom projektgruppen sker i enlighet med gruppkontraktet (se appendix).

\subsection{Definition av arbetsinnehåll och ansvar}



%Tabell med leveranser
\begin{center}
\begin{longtable}{|l |p{.8\linewidth}|} \hline

Projektledare & 
Adnan Berberovic \\ \hline

Dokumentansvarig & 
Person \\ \hline

Testansvarig & 
Person \\ \hline


Designansvarig hårdvara &
Person \\ \hline


Designansvarig mjukvara & 
Person \\ \hline
\end{longtable}
\end{center}

Leveransansvarig 
%Ange alla inblandade personer och deras ansvarsområden.
%Definiera arbetsinnehållet för projektets roller.
\pagebreak

\section{Dokumentplan}

\pagebreak

\section{Utvecklingsmetodik}

\pagebreak

\section{Utbildningsplan}

\subsection{Egen utbildning}

\pagebreak

\section{Rapporteringsplan}

\pagebreak

\section{Mötesplan}

\pagebreak

\section{Resursplan}

\subsection{Personer}

\subsection{Material}

\subsection{Lokaler}

\subsection{Ekonomi}

\pagebreak

\section{Milstolpar och beslutspunkter}

\subsection{Milstolpar}

\subsection{Beslutspunkter}

\pagebreak

\section{Aktiviteter}

\pagebreak

\section{Tidplan}

\pagebreak

\section{Kvalitetsplan}

\subsection{Granskningar}

\subsection{Testplan}

\pagebreak

\section{Prioriteringar}

\pagebreak

\section{Projektavslut}

\setcounter{secnumdepth}{0}
\pagebreak
\section{Referenser}
Kravspecifikation för TSEA56 2015, grupp 2\\
kravspec\_v1.0.pdf \\[0.1in]

LIPS

\setcounter{secnumdepth}{2}


\end{flushleft}



\end{document}
