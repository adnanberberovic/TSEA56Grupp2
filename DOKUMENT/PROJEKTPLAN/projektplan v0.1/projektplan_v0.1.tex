\documentclass[11pt]{article}
\usepackage[margin=3cm]{geometry}
\usepackage[swedish]{babel}
\usepackage[utf8]{inputenc}
\usepackage[T1]{fontenc}
\usepackage{fancyhdr}
\usepackage{ragged2e}
\usepackage{titling}
\usepackage{graphicx}
\usepackage{pbox}
\usepackage{tabularx}
\usepackage{longtable}
\usepackage{tabu}
\usepackage{url}

\graphicspath{ {images/} }
\newcommand{\subtitle}[1]{%
  \posttitle{%
    \par\end{center}
    \begin{center}\large#1\end{center}
    \vskip0.5em}%
}
\newcounter{kravc}
\setcounter{kravc}{1}
\newcommand{\kravcc}{
	\thekravc
	\stepcounter{kravc}
}
\newcounter{refc}
\setcounter{refc}{1}
\newcommand{\reff}{
	\therefc
	\stepcounter{refc}
}
\pagestyle{fancy}


\date{}
\pagenumbering{roman}
\chead{Undsättningsrobot}
\rhead{2015-02-11}
\lhead{}
\lfoot{
	TSEA56 
	\\
	Projektplan
}
\rfoot{
	Projektgrupp 2
	\\
	e-post: adnbe196@student.liu.se
}

\begin{document}
\begin{titlepage}
\begin{center}
{\Large\bfseries TSEA56 - Kandidatprojekt i elektronik \\ LIPS Projektplan}\\
%
\vspace{2\baselineskip}
%
Version 0.1\\
\vspace{2\baselineskip}
%
Grupp 2 \\
Agafonov, Nikolaj, 
\texttt{nikag669}
\\
Berberovic, Adnan, 
\texttt{adnbe196}
\\
Brorsson, Andreas, 
\texttt{andbr981}
\\
Fridborn, Fredrik, 
\texttt{frefr166}
\\
Oprea, Robert, 
\texttt{robop806}
\\
Skytt, Måns, 
\texttt{mansk700}

\vspace{2\baselineskip}
2015-02-11

\vspace{25\baselineskip}
Status
\begin{longtable}{|l|l|l|} \hline

Granskad &
 & 
 \\ \hline
Godkänd &
 &
 \\ \hline
 
\end{longtable}

\end{center}
\end{titlepage}

\pagebreak
\begin{center}

\section*{PROJEKTIDENTITET}
2015/VT, Undsättningsrobot Gr. 2
\\
Linköpings tekniska högskola, ISY
\\[0.5in]
\begin{table}[h]
\begin{tabular}{|l|l|l|l|} \hline
Namn & Ansvar & Telefon & E-post \\[0.1in] \hline
Nikolaj Agafonov & Dokumentansvarig (DA) & 072-276 99 46 & nikag669@student.liu.se \\ \hline
Adnan Berberovic & Projektledare (PL) & 070-491 96 07 & adnbe196@student.liu.se \\ \hline
Andreas Brorsson & Leveransansvarig (LA) & 073-524 44 60 & andbr981@student.liu.se \\ \hline
Fredrik Fridborn & Designansvarig mjukvara (DAM) & 073-585 52 01 & frefr166@student.liu.se \\ \hline
Robert Oprea & Testansvarig (TA) & 070-022 10 18 & robop806@student.liu.se \\ \hline
Måns Skytt & Designansvarig hårdvara (DAH) & 070-354 28 84 & mansk700@student.liu.se \\ \hline
\end{tabular}
\end{table}

E-postlista för hela gruppen: adnbe196@student.liu.se
\\[1in]
Kund: Kent Palmkvist, 581 83 Linköping,
Kundtelefon: 013-28 13 47, kentp@isy.liu.se
\\[1in]
Kursansvarig: Tomas Svensson, 013-28 13 68, tomass@isy.liu.se
\\
Handledare: Olov Andersson, 013-28 26 58, Olov.Andersson@liu.se
\end{center}
\pagebreak

\tableofcontents

\pagebreak

\section*{Dokumenthistorik}
\begin{table}[h]
\begin{tabular}{|l|l|l|l|l|} \hline

Version & 
Datum & 
Utförda förändringar & 
Utförda av & 
Granskad \\[0.1in] \hline
0.1 &
2015-02-11 & 
Första utkastet & 
Grupp 2 & 
\\ \hline

\end{tabular}
\end{table}


\pagebreak

\pagenumbering{arabic}

\begin{flushleft}
\section{Beställare}
Projektets beställare är Kent Palmkvist som representerar ISY.

\section{Översiktlig beskrivning av projektet}

\subsection{Syfte och mål}
Mål är att leverera en produkt, en robot, som kan köra autonomt och via fjärrstyrning i okända, möjligtvis farliga, miljöer. Dessutom ska projektet visa hur man tillämpar kunskap från de kurser man läst, samt ge erfarenhet i projektarbete och förståelse för hur hårdvara och mjukvara interagerar.
\subsection{Leveranser} %Lägga till aktiviteter?
Leveranser skall göras senast på nedan nämnda tider och datum om inte annat är överenskommet mellan beställare och projektgrupp.

%Tabell med leveranser
\begin{center}
\begin{longtable}{|l |p{.8\linewidth}|} \hline

3 feb: & 
kl 16.00: Kravspecifikationen ska vara klar. (BP1) \\ \hline

16 feb: & 
kl 16.00: Första versionen av projektplan, tidplan och systemskiss ska vara inlämnade till beställaren. \\ \hline

20 feb: & 
kl 16.00: Slutgiltig version av projektplan, tidplan och systemskiss ska vara inlämnade till beställaren. \\ \hline

5 mars: &
kl 16.00: första version av förstudien (minst 5 sidor) ska skickas till respektive handledare och till er beställare. \\ \hline

11 mars: & 
kl 16.00: Första versionen av designspecifikationen ska vara inlämnad till handledaren. \\ \hline

24 mars: &
Designspecifikationen ska vara godkänd av handledaren vid ett beslutsmöte BP3. \\ \hline

1 april: &
kl 16:00 Version 1.0 av förstudien ska skickas till respektive handledare och till beställare. \\ \hline

17 april: & 
Nuvarande design ska vara presenterad för och godkänd av handledaren vid ett beslutsmöte BP4. \\ \hline

25 maj: &
Verifiering av kraven (BP5) bör ske i god tid innan redovisningen. Utan detta beslut får ni inte leverera! \\ \hline

21 maj: &
Kappan, version 1.0, (exklusive appendix) ska levereras. Se nedan. \\ \hline

27 maj: &
Teknisk dokumentation och användarhandledning (båda version 1.0) ska vara inlämnade. Slutversion av skrivarbete skall också skickas med vid detta tillfälle. \\ \hline

Vecka 23: &
Redovisning och demonstration.\\ \hline

2 juni: &
(preliminärt) 8.15-17 muntliga presentationer och opposition. Tider se nedan. \\ \hline

3 juni: &
(preliminärt) 9.15-17 tävlingar utanför café Java. \\ \hline

5 juni: &
Efterstudien ska vara inlämnad. Vid denna tidpunkt ska även källkod skickas in i en zip-fil. \\ \hline

12 juni: &
Bärbar dator och övrig utrustning ska vara återlämnade. \\ \hline
\end{longtable}
\end{center}

En tidrapport ska lämnas senast kl 16.00 vid följande datum: 4 febr, 23 febr, 9 mars, 23 mars, 30 mars, 13 april, 20 april, 27 april, 4 maj, 11 maj, 18 maj, 25 maj, 1 juni och 8 juni.

\subsection{Begränsningar} %Vad behöver vi inte göra?
Miljön som roboten körs i är begränsad på så sätt att den max ska vara 6x6m och passager måste vara minst 40 cm breda. Inga blockerande hinder får förekomma i vägen för roboten, den kan endast röra sig på släta ytor.


\section{Fasplan}
Nedan ges en grov beskrivning av aktiviteterna i varje fas.
\subsection{Under projektet}
Initialt kommer en stor del av arbetet bestå av att lära sig de verktyg som behövs för att kunna genomföra projektet: AVR, VDHL, dataöverföring via bluetooth, mätteknik etc. Det sker dels genom läsning men även laboratoriskt. Därefter kommer hårdvaran konstrueras och testning kommer ske kontinuerligt. Efter varje modul skapats kommer större tester genomföras. Slutligen kommer mjukvaran kodas med kontinuerlig testning.

\subsection{Efter projektet}
En färdig produkt ska levereras. Efter projektet kommer efterstudie genomföras, labbplatsen städas och materiel återlämnas varpå projektgruppen upplöses och gruppkontraktet hävs. Gruppen kommer även att reflektera över ett utfört projektarbete och kunna föreslå förbättringar.
\pagebreak

\section{Organisationsplan för hela projektet} 

\subsection{Villkor för samarbetet inom projektgruppen}
Samarbetet inom projektgruppen sker i enlighet med gruppkontraktet (se appendix).

\subsection{Definition av arbetsinnehåll och ansvar}

\begin{tabu} to 1.05\textwidth { | X[c] | X[c] | X[c] | }
 \hline
 Projektledare & Adnan Berberovic & Ansvarig för projektgruppen \\
 \hline
 Dokumentansvarig & Person & Ansvarig för dokument\\
 \hline
Testansvarig & Person & Ansvarig för testning \\ 
\hline
Designansvarig hårdvara & Person  & Ansvarig för hårdvarudesign\\
\hline
Designansvarig mjukvara & Person  & Ansvarig för mjukvarudesign\\
\hline
Leveransansvarig & Person & Ansvarig för leveranser \\
\hline
 \end{tabu}

%Ange alla inblandade personer och deras ansvarsområden.
%Definiera arbetsinnehållet för projektets roller.

\pagebreak

\section{Dokumentplan}
%Lista alla dokument som ska produceras i tabellen.
%Ange ansvarig, vem som godkänner, syftet, vem de
%ska distribueras till och när dokumentet ska vara klart. 
Följande tabell räknar upp de dokument som kommer att skapas under projektets gång, syftet, vem som är ansvarig, vem som godkänner, vem de ska distribueras till samt när dokumentet ska vara klart.
\begin{center}
\begin{longtable}{|p{.24\linewidth}|p{.08\linewidth}|p{.25\linewidth}|p{.19\linewidth}|p{.15\linewidth}|}\hline
\textbf{Dokument} & \textbf{Syfte} & \textbf{Ansvarig} & \textbf{Godkännare} & \textbf{Målgrupp} \\ \hline

Kravspecifikation & SE & Listar alla krav som slutprodukten ska uppfylla. & Projektgrupp och beställare & .pdf \\ \hline
Projektplan & SE & Beskriver hur projektet ska utföras & Projektgrupp & .pdf \\ \hline
Tidplan & SE & Beskriver när aktiviteter ska utföras och av vem & Projektgrupp & .xls \\ \hline
Systemskiss & SE & Beskriver hur produkten ska konstrueras& Projektgrupp och beställare & .pdf \\ \hline
Förstudie & SE & Analysera huruvida projektet kan drivas framåt eller inte & Projektgrupp & .pdf \\ \hline
Design-specifikation & SE & Beskriver mer detaljerat hur produkten ska konstrueras & Projektgrupp & .pdf \\ \hline
Kappa & SE & Sammanfattar alla dokument som beställaren kan vara intresserad av & Beställare & .pdf \\ \hline
Teknisk dokumentation & SE & Beskriver hur produkten fungerar & Beställare & .pdf \\ \hline
Användar-handledning & SE & Beskriver hur man använder produkten& Beställare & .pdf \\ \hline
Efterstudie & SE & En reflektion kring hur projektet bedrevs. Vad kunde man ha gjort bättre, etc.& Projektgrupp & .pdf \\ \hline

\end{longtable}
\end{center}
\pagebreak

\section{Utvecklingsmetodik}
Arbetet kommer att delas upp mellan gruppmedlemmarna på så sätt att projektgrupp kommer att bestå av mindre grupper (exempelvis grupper om två eller tre personer). Uppdelningen beror på uppgiftens svårighet och tidsåtgång. En sådan uppdelning är tänkt att förbättra och snabba upp utförandet av projektet. Projektets delar som implementeras var för sig måste kunna fungera tillsammans med de andra delar, därför ska de mindre projektgrupper komma överens om olika delars detaljer och veta hur det hela systemet ska fungera.


Tänker ni använda er av någon speciell metodik?
Alltid jobba i par? Använda ett visst
programspråk?


\pagebreak

\section{Utbildningsplan}


\subsection{Egen utbildning}
För att kunna implementera och testa systemets olika komponenter, både mjukvara och hårdvara, behöver gruppen att inhämta kunskap om de program och mätverktyg, som är relevanta för projektet. Exempelvis kommer gruppen att kunna använda utvecklingssystemet AVR-Studio och debugverktyget JTAGICE för att programmera de AVR-processorer som kommer att finnas i varje delmodul. Projektgruppen kommer att lära sig att programmera kretsar med VHDL-programmeringsspråk samt utföra mätningar med en logikanalysator.

\pagebreak

\section{Rapporteringsplan}
Vid bestämda datum, ungefär varje vecka, kommer projektledaren att rapportera den tid som gruppen har spenderat fram till rapporteringen. Tidsrapporten uppdateras löpande av alla gruppmedlemmar. Till tidrapporteringen kommer även en statusrapport att skickas med, som beskriver:\\
$\bullet$ \textit{Vilka framsteg har gjorts sedan förra tidrapporteringen?}\\
$\bullet$ \textit{Finns det några problem?}\\ 
$\bullet$ \textit{Vad ska göras under kommande veckan?}


\section{Mötesplan}
Projektgruppen kommer att träffas löpande under projektet för avstämningar mot tidplanen, samt planera kommande dagar. Möten kommer att ske 1 gång i veckan. Extra möten kan tillkomma, exempelvis möte med handledare och beställare.


\section{Resursplan}
\subsection{Personer}
Till projektgruppens förfogande kommer det att finnas en handledare tillgänglig som hjälp om det så behövs.

\subsection{Material}
Projektgruppen har till förfogande ett robotchassi och ett antal sensorer som finns beskrivna på Vanheden$^{[\reff]}$, ISY:s datablad.

\subsection{Lokaler}
Projektgruppen kommer att ha tillgång till laborationssalen MUXEN, där större delen av projektets tid kommer att spenderas. Projektgruppen kommer även att vid gruppmöten och/eller dokumentering att utnyttja till exempel ISYtan:s grupprum.

\subsection{Ekonomi}
Projektgruppen har tillgång till 1380 timmar totalt arbete och labutrustning i laborationssal MUXEN. Projektgruppen har inga finansiella tillgångar.

\pagebreak

\section{Milstolpar och beslutspunkter}


\subsection{Milstolpar}
\begin{table}[h]
\begin{tabular}{|l|p{.75\linewidth}|l|} \hline

Nr &
Beskrivning &
Datum \\ \hline

1 &
Fungerande sensorsystem &
 \\ \hline
2 &
Fungerande reglersystem &
 \\ \hline
3 &
Fungerande kommunikationssystem &
 \\ \hline
4 &
Fungerande kartläggningsalgoritm &
 \\ \hline
5 &
Fungerande optimeringsalgoritm för kortast väg &
 \\ \hline
6 &
... &
 \\ \hline
7 &
Färdig robot &
 \\ \hline
 
\end{tabular}
\end{table}

\subsection{Beslutspunkter}
\begin{table}[h]
\begin{tabular}{|l|p{.75\linewidth}|l|} \hline

Nr &
Beskrivning &
Datum \\ \hline

0 &
Godkännande av uppdrag, beslut att skriva kravspecifikation &
2015-01-23 \\ \hline
1 &
Godkännande av kravspecifikation, beslut att göra projektplan, systemskiss &
2015-02-03 \\ \hline
2 &
Godkännande av projektplan och systemskiss, beslut att påbörja under-fasen &
2015-02-20 \\ \hline
3 &
Godkännande av designspecifikation, beslut att påbörja konstruktion &
2015-03-24 \\ \hline
4 &
Godkännande av nuvarande design &
2015-04-17 \\ \hline
5 &
Verifiering av kravspecifikationen, beslut att leverera och påbörja efterfasen &
2015-05-25 \\ \hline
6 &
Godkännande av slutrapport, beslut att upplösa projektgruppen &
2015-06-05 \\ \hline
 
\end{tabular}
\end{table}

\pagebreak

\section{Aktiviteter}
\begin{table}[h]
\begin{tabular}{|l|p{.30\linewidth}|l|p{.40\linewidth}|p{.10\linewidth}|} \hline

Nr & 
Aktivitet & 
Ansvar & 
Beskrivning & 
Beräknad total tid \\[0.1in] \hline




\end{tabular}
\end{table}


\section{Tidplan}
Se bifogat dokument \textit{tidplan\_v0.1}.

\pagebreak

\section{Kvalitetsplan}
För att se till att minska på problematiska händelser under projektets gång kommer vi att vidta åtgärder som kodgranskning och hårdvarutester. Dessa förklaras närmare i de kommande delsektionerna.

\subsection{Granskningar}
Kod ska granskas på så sätt att de följer en kodkonvention som gruppen har kommit överens om.\\
Dokument granskas så tekniska och språkliga begrepp används korrekt och att formateringar på dokumentens innehåll inte är fel.

\subsection{Testplan}
Tester kommer att utföras löpande under projektets gång. Varje delkomponent kommer att testas för sig. När en funktion är färdig testas den och arbetet går vidare till nästa problem.

\pagebreak

\section{Prioriteringar}


\pagebreak

\section{Projektavslut}
Projektet kommer att avslutas med en avstämning mot alla krav och dokumentationer. Dessutom kommer en redovisning och demonstration av projektet att ske vecka 23. När allt är godkänt upphör gruppkontraktet och projektgruppen upplöses.
\\[0.1in]



\setcounter{secnumdepth}{0}
\pagebreak
\section{Referenser}
Kravspecifikation för TSEA56 2015, grupp 2\\
kravspec\_v1.0.pdf \\[0.1in]

LIPS \\[0.1in]

[1] Vanheden, ISY:s datablad: \url{https://docs.isy.liu.se/twiki/bin/view/VanHeden}

\setcounter{secnumdepth}{2}


\end{flushleft}



\end{document}
