% LaTeX Document Template v1.0.1
% Made by Adnan Berberovic
% Contact:
%	E-mail: adnbe196@student.liu.se

\documentclass[12pt]{article}
\usepackage[margin=3cm]{geometry}
\usepackage[swedish]{babel}
\usepackage[utf8]{inputenc}
\usepackage[T1]{fontenc}
\usepackage{fancyhdr}
\usepackage{ragged2e}
\usepackage{titling} % Make custom titles.
\usepackage{graphicx} % Allow graphics.
\usepackage{pbox} % Used for margins in tables.
\usepackage{tabularx} % Used for tables.
\usepackage{longtable} % Used for tables.
\usepackage{tabu} % Used for tables.
\usepackage{url} % Allow urls.
\usepackage{amsmath} % Allow math \begin{equation}, end
\usepackage[backend=bibtex,style=ieee]{biblatex} % Bibliography package. Use 'biber' instead of 'bibtex' if you're using ShareLaTeX. This is dependant on your default bibliography tool.
\usepackage{csquotes} % Spawns an error if not included. Not sure why.
\usepackage[hidelinks]{hyperref} % To include urls, hyper links and clickable references
\usepackage[nottoc]{tocbibind} % To include bibliography in ToC
\usepackage{tikz} % Include tikzpicture

\pagestyle{fancy}
\date{} % Sets date to blank. Use \today to print todays date.

\pagenumbering{roman} % Set roman page numbering before the actual document begins.
\chead{Regleruppgift 2} % Replace argument with the type or title of your report.
\rhead{\today} % Set date here.
\lhead{} % Put anything here.
\lfoot{TSEA56\\ISY} % Replace COURSE with the course code, INSTITUTION with respective institution.
\rfoot{adnbe196 \\ robop806} % Replace AUTHOR with your name. Add more text if you deem it necessary.

\graphicspath{ {images/} } % Set graphics path if images are to be used in document

\addbibresource{refs.bib} % Import reference bibliography

% Use this command, and make more if you need, to keep some kind of numbering order,
% without the need to replace numbers if something in your list needs to be removed/added.
% To make more, just copy and paste the following 6 lines and replace "cter" with the counter
% name, ccounter with the command name you wish to use. (Remember to replace \thecter with
% \the"newname").
\newcounter{cter}
\setcounter{cter}{1}
\newcommand{\ccounter}{
	\thecter.
	\stepcounter{cter}
}

\begin{document}
	\begin{titlepage}
		\begin{center}
			
			{\Large\bfseries Förstudie} \\ % Replace "NAME OF DOCUMENT" with... you know.
			\vspace{0.5\baselineskip}
			Regleruppgift 2\\
			\vspace{0.5\baselineskip}
			Version 0.1
			
			\vspace{2\baselineskip}
			
			Adnan Berberovic, \texttt{adnbe196}, 931229-5570 \\
		    Robert Oprea, \texttt{robop806}, 930508-9618 \\ % Replace AUTHOR with your name and autho123 with your ID.
		    
		    \vspace{4\baselineskip}
		    Beställare: Kent Palmkvist\\
		    Handledare ISY: Johan Löfberg\\
		    Handledare TEMA: Birgitte Saxstrup\\
			
			\vspace{3\baselineskip}
			
			\today	% Replace with date.
			
		\end{center}
	\end{titlepage}
	
	\setcounter{secnumdepth}{0} % Prevent section numbering of the following sections.
	\section*{Sammanfattning}
	Denna förstudie behandlar frågor som rör kartläggning, lokalisering och optimeringsproblem för kortaste väg och kopplas till kandidatprojektet i elektronik där svaren på problemen kommer att vara av ytterst relevans.\\[0.1in]
	Kartläggning och lokalisering kommer att beskrivas av en metod kallad \textbf{SLAM}, \textit{(simultaneous localisation and mapping)}. Problemet löses med stor valfrihet beroende av tillgängliga sensorer, beräkningskraft och miljön i fråga.\\[0.1in]
	Optimeringsproblemet är av typen billigaste väg-problem och kommer att behandlas med \textit{dijkstras algoritm}. Denna algoritm går ut på att i befintlig position hitta den billigaste vägen till den position som är av intresse.
	
	\pagebreak
	
	\setcounter{secnumdepth}{3} % Resume numbering of the following sections.
	
	\tableofcontents	% Make a ToC.
	
	\pagebreak
	
	
	\pagenumbering{arabic}	% Start page numbering with arabic numbers, for the rest of the document.
	
	\section{Inledning}
	Denna förstudie ämnar ge fördjupad kunskap inom några av de delar som ska implementeras i den robot som ska framtas i kandidatprojektet i elektronik. Roboten i fråga ska effektivt kunna avsöka och kartlägga en okänd omgivning, hitta ett bestämt mål, och sedan via den kortaste vägen navigera tillbaka till ursprungspositionen. Detta ställer många krav på robotens funktionalitet, och den ska klara av många olika moment. För att projektet ska kunna genomföras på bästa sätt utförs tre olika förstudier som ämnar fördjupa projektgruppens kunskaper inom olika områden, så att de har en god kunskapsgrund att stå på när val roboten ska byggas.\\[0.1in]
	
	\section{Problemformulering}
	\label{sec:problemform}
	De problem som behandlas i denna studie är följande:
	\begin{description}
	     \item[Problem 1:] Avsökning och kartläggning av en omgivning
	     \item[Problem 2:] Beräkning av optimal väg
	\end{description}
	I samband med att dessa problem studeras kommer vissa metoder som löser dessa att beskrivas något ingående, tillräckligt för att kunna dra nytta av detta till kandidatprojektet i fråga.\\[0.1in]
	%Hur kan man söka av ett rum/labyrint/omgivning på ett effektivt sätt? Beskriv olika kartläggningsalgoritmer. Hur beräknas den kortaste vägen till en punkt?
	
	\section{Kunskapsbas}
	Som källor har boken \textit{Optimeringslära} \autocite{lund-03}, samt artikeln \textit{Improved Techniques for Grid Mapping with Rao-Blackwellized Particle Filters} \autocite{gris-07} använts. Boken kommer från studentlitteratur, och artikeln fanns på bibliotekets databas.%\\[0.1in]
	
	\pagebreak
	
	\section{Avsökning och kartläggning av en omgivning} 
	
	
	Problemet som betraktas här är att en autonom robot ska utforska och avsöka en okänd omgivning, samtidigt som den håller reda på sin egen position. Detta problem benämns allmänt som \textit{simultaneous localisation and mapping} (SLAM). Allmänt kan SLAM-problemet delas upp i mindre delproblem såsom identifierandet av landmärken, dataassociation, uppskattning och uppdatering av position samt uppdatering av landmärken. Dessa delproblem kan lösas med stor valfrihet beroende på tillgängliga sensorer, beräkningskraft och speciellt den miljön man ämnar avsöka. Framöver kommer fokus läggas på avsökning i en inomhusmiljö bestående av ett rutnät. \autocite{gris-07}
	
	
	\subsection{Problemformulering}
	Generellt kan SLAM-problemet beskrivas på följande sätt:
	\begin{equation}
	p(m_{t},x_{t}|o_{1:t})
	\end{equation}
	där målet är att beräkna positionen $x_{t}$ samt kartan $m_{t}$ givet sensorobservationerna $o_{t}$ under diskreta tidssteg $t$. Med hjälp av Bayes sats fås följande:
	\begin{equation}
	\left\{\begin{matrix}
        p(x_{t}|o_{1:t},m_{t})\\
        p(m_{t}|x_{t},o_{1:t}).
        \end{matrix}\right.
	\end{equation}
	I (2) uppstår en algebraisk loop där robotens position samt kartan beror av varandra. För att lösa detta problem används olika statistiska metoder, t.ex. kalmanfilter eller partikelfilter. Metoden som här kommer beskrivas använder sig av \textit{Rao-Blackwellized partikelfilter} (RBPF). Den utgår från följande problem:
	\begin{equation}
	p(x_{1:t},m|z_{1:t},u_{1:t-1})
	\end{equation}
där $m$ är kartan, $x_{1:t}=x_{1},...,x_{t}$ är robotens rörelsebana,  $z_{1:t}=z_{1},...,z_{t}$ är sensorobservationer och $u_{1:t-1}=u_{1},...,u_{t-1}$ är distansmätningarna utförda av roboten.

    \subsection{Kartläggningsalgoritm}
    För att beräkna kartan skrivs (3) om som
        \begin{equation}
	    p(m|x_{1:t},z_{1:t})\cdot p(x_{1:t}|z_{1:t},u_{1:t-1})
	    \end{equation}
    vilket tillåter att robotens bana beräknas först, och sedan beräknas kartan med hjälp av denna.
    
    \subsubsection{Beräkning av robotens position}
    För att beräkna robotens position kan ett partikelfilter appliceras. Varje partikel representerar en möjlig bana, och en karta associeras till varje partikel. Varje karta är framtagen utifrån mätdata samt den associerade banan. Den partikelfilteralgoritm som används är Rao-Blackwellized \textit{sampling importance resampling} (SIR). Denna algoritm kan sammanfattas i följande steg:
    \begin{description}
    \item[1. Sampling:] Nästa mängd partiklar $\{x_{t}^{(i)}\}$ fås genom sampling ur den föreslagna distributionen (eng. proposal distribution) $\pi$.
    \item[2. Viktning:] Varje partikel tilldelas en vikt $w_{t}^{(i)}$ enligt
        \begin{equation}
        w_{t}^{(i)}=\frac{p(x_{1:t}^{(i)}|z_{1:t},u_{1:t-1})}{\pi(x_{1:t}^{(i)}|z_{1:t},u_{1:t-1})}.
        \end{equation}
    \item[3. Omsampling:] Nya partiklar väljs ut med sannolikhet proportionell mot deras vikt. Efter omsampling har alla partiklar samma vikt.
    \item[4. Kartskattning:] För varje partikel beräknas en skattning av motsvarande karta $p(m^{(i)}|x_{1:t}^{(i)},z_{1:t})$ givet banan $x_{1:t}^{(i)}$ och tidigare sensorobservationer $z_{1:t}$.
	\end{description}	
		
	\pagebreak
	
	\section{Beräkning av optimal väg}
	Billigaste väg-problem är ett av de mest grundläggande problem som förekommer inom nätverksoptimering. Problemet som analyseras är att hitta den kortaste vägen från en punkt till en annan i ett känt eller delvis känt område. \autocite{lund-03}
	
	\subsection{Definitioner}
	För att lösa och förstå detta problem behövs några definitioner.
	\begin{itemize}
		\item En \textit{nod} är en tillståndspunkt. Betecknas $n_{i}$ för något $i$.
		\item En \textit{båge} är en väg mellan två noder. Dess kostnad betecknas med $c_{ij}$, dvs kostnaden från $n_{i}$ till $n_{j}$.
		\item En \textit{riktad} båge är en båge med bestämd riktning. 
		\item En \textit{cykel} är en uppsättning bågar som ger upphov till att en nod kan besökas flera gånger.
		\item \textit{Nodpris} är kostnaden att ta sig till noden i fråga. Betecknas $y_{i}$.
		\item \textit{Föregångare} är noden som besöktes precis innan nuvarande nod. Betecknas $p_{i}$.
	\end{itemize}
	
	\subsection{Egenskaper hos billigaste väg-problem}
	I nätverksoptimering där en lösning av billigaste väg-problemet är önskat krävs att vissa förutsättningar uppfylls på nätverket. \autocite{lund-03}
	\begin{itemize}
	    \item Endast riktade bågar förekommer.
	    \item Nod $n_{t}$ är uppnåelig från nod $n_{s}$.
	    \item Inga cykler med \textit{negativ kostnad} förekommer.
	\end{itemize}
	Om en oriktad båge skulle förekomma ersätts den med två riktade bågar mellan noderna med samma kostnad.\\
	Om nod $n_{t}$ inte är uppnåelig från nod $n_{s}$ saknar problemet lösning.\\
	Om det finns cykler med en negativ kostnad har problemet en obegränsad lösning. \autocite{lund-03}\\
	Detta är i enlighet med det optimeringsproblem som uppstår i kandidatprojektet.
	
	\subsection{Bellmans ekvationer}
	Ett billigaste väg-problem kan formuleras med hjälp utav \textit{Bellmans ekvationer}. När vi har ett känt antal, $n$, noder i nätverket kan ekvationerna skrivas:
	\begin{equation}
	    \left\{\begin{matrix}
        y_{s} = 0\\
        y_{j} = \min\limits_{i:(i,j) \in B} y_{i} + c_{ij}, \quad j = 2, .., n
        \end{matrix}\right.
	\end{equation}
	där $B$ är mängden av alla bågar i nätverket, $c_{ij}$ är kostnaden på båge $(i,j)$ och $y_{j}$ är den billigaste vägen från startnoden $n_{s}$ till nod $n_{j}$. \autocite{lund-03}\\
	Alla algoritmer som löser ett billigaste väg-problem försöker att systematiskt uppfylla Bellmans ekvationer.
	
	%\pagebreak
	
	\subsection{Dijkstras algoritm}
    Dijkstras algoritm är en algoritm som löser just det beskrivna problemet. Denna algoritm bestämmer alltså en billigaste väg från en nod, $n_{s}$ till en annan nod $n_{t}$, i ett nätverk med nodmängden N och bågmängden B. \\
    Algoritmen kan sammanfattas enligt följande: \autocite{lund-03}
    \begin{description}
        \item[Steg 0] Dela upp nodmängden i två mängder, Avsökta noder, $A = \emptyset$, och ej avsökta noder, $D = N$. Initiera startnoden med pris $y_{s} = 0$ och alla övriga noder med initialt nodpris $y_{j} = \infty$\\
        Märk startnoden med $(p_{s}, y_{s}) = (-, 0)$ och övriga noder med $(-, \infty)$.
        \item[Steg 1] Sök efter den billigaste noden $n_{i} \in D$.
        \item[Steg 2] Avsök nod $n_{i}$, det vill säga undersök alla bågar $(i,j) \in B$ utgående från nod $n_{i}$. Om totalkostnaden fram till nuvarande nod $n_{i}$ och bågkostnaden till nod $n_{j}$ tillsammans blir mindre än nodpriset $y_{j}$ så finns en billigare väg från $n_{s}$ till $n_{j}$ som går igenom $n_{i}$ hittats. Märk $n_{j}$ med nya värden, $(p_{j}, y_{j}) = (i, y_{i} + c_{ij})$.
        \item[Steg 3] Flytta över den avsöka noden $n_{i}$ från mängden D till A.
        \item[Steg 4] Om alla noder har avsökts har en billigaste väg funnits. Annars gå till Steg 1 och repetera.
    \end{description}
    
    \subsubsection{Komplexitet}
    I dijkstras algoritm kommer avsökningen av en nod att, i värsta fall, innebära att n-1 bågar undersöks, på totalt n noder. Detta innebär att komplexiteten uppskattas till att vara proportionell mot $n^{2}$. \autocite{lund-03}
	
	\section{Slutsats}
	De problem som beskrevs i kapitel \ref{sec:problemform}~Problemformulering och metoder till att lösa dessa problem har diskuterats i den utsträckning att lösningarna är tillämpningsbara till det kandidatprojekt denna studie är avsedd för.
	
	\subsection{Avsökning och kartläggning av en omgivning}
	Ingen slutsatts har ännu dragits.
	
	\subsection{Beräkning av optimal väg}
	Den optimeringsalgoritm som har valts är tillräcklig för att lösa det optimeringsproblem som uppkommer i kandidatprojektet. Den väg som ska optimeras är inte tillräckligt stor för att komplexiteten ska medföra någon märkbar skillnad i tidsåtgång, dessutom kommer man alltid fram till kortaste vägen. Denna metod är också till synes inte allt för komplicerad att implementera, varför denna metod har valts som lösning till optimeringsproblemet i fråga.
	
	\pagebreak
		
	\setcounter{secnumdepth}{0} % Prevent page numbering yet again. 
	\addcontentsline{toc}{section}{Referenser} % Add reference to ToC. Might have to compile TWICE.
	
	\printbibliography % Prints bibliography. Look at refs.bib file in the same directory and make sure you have all
	% references at hand when autociting them.
\end{document}
	
	